\chapter{Dnevnik promjena dokumentacije}
		
		\textbf{\textit{Kontinuirano osvježavanje}}\\
				
		
		\begin{longtabu} to \textwidth {|X[2, l]|X[13, l]|X[3, l]|X[3, l]|}
			\hline \multicolumn{1}{|l|}{\textbf{Rev.}}	& \multicolumn{1}{l|}{\textbf{Opis promjene/dodatka}} & \multicolumn{1}{|l|}{\textbf{Autori}} & \multicolumn{1}{l|}{\textbf{Datum}} \\[3pt] \hline
			\endfirsthead
			
			\hline \multicolumn{1}{|l|}{\textbf{Rev.}}	& \multicolumn{1}{l|}{\textbf{Opis promjene/dodatka}} & \multicolumn{1}{|l|}{\textbf{Autori}} & \multicolumn{1}{l|}{\textbf{Datum}} \\[3pt] \hline
			\endhead
			
			\hline 
			\endlastfoot
			
			0.1 & Napravljen predložak.	& Nuić & 15.10.2020. 		\\[3pt] \hline 
			
			0.2 & Dodani dionici i funkcionalni zahtjevi trenera.	& Marfat & 16.10.2020. 		\\[3pt] \hline 
			
			0.3 & Dodani funkcionalni zahtjevi neprijavljenog korisnika i sportaša i dodan dio obrazaca uporabe.	& Rašić & 16.10.2020. 		\\[3pt] \hline 
			
			0.4 & Uređeni funkcionalni zahtjevi trenera zbog konzistentnosti, dodani funkcionalni zahtjevi baze podataka i dio obrazaca uporabe. & Marfat & 17.10.2020. \\[3pt]
			\hline
			
			0.5 & Dodani funkcionalni zahtjevi i obrasci uporabe za iznajmljivača i administratora. & Ilić & 17.10.2020. \\[3pt]
			\hline
			
			0.6 & Dodani ostali zahtjevi i glavni sudionik u UC20 promijenjen u "Trener". 
			\newline Dodan sekvencijski dijagram za UC6 i postavljen defaultni direktorij za slike u folder "slike". 
			\newline Dodani ostali sekvencijski dijagrami. & Marfat & 31.10.2020. \\[3pt]
			\hline
			
			0.7 & Promijenjeno "Korisnik" u "Sportaš" u obrascima uporabe nakon prijave & Crnogorac & 31.10.2020. \\[3pt]
			\hline
			
			0.8 & U opisu projekta dodana postojeća slična rješenja i moguće nadogradnje projektnog zadatka	& Rašić & 31.10.2020. 		\\[3pt] \hline
			
			0.9 & Dodani UML dijagrami u PNG formatu u direktorij slike te u LaTeX dokument	& Crnogorac & 31.10.2020. 		\\[3pt] \hline
			
			0.10 & Dodan opis projektnog zadatka	& Paradžik & 1.11.2020. 		\\[3pt] \hline
			
			0.11 & Dodan opis baze podataka i relacija	& Srdarev & 2.11.2020. 		\\[3pt] \hline
			
			0.12 & Dodan dijagram baze podataka	& Srdarev & 4.11.2020. 		\\[3pt] \hline
			
			0.13 & Promijenjen UC10	& Marfat & 5.11.2020. 		\\[3pt] \hline
			
			0.14 & Ažuriran dnevnik sastajanja	& Rašić & 8.11.2020 		\\[3pt] \hline 
			
			0.15 & Restrukturirana baza podataka	& Srdarev & 8.11.2020. 		\\[3pt] \hline
			
			0.16 & Ispravak gramatičkih i pravopisnih grešaka	& Paradžik & 8.11.2020 		\\[3pt] \hline 
			
			0.17 & Ažuriran dnevnik sastajanja	& Rašić & 13.11.2020 		\\[3pt] \hline 
			
			0.18 & Uređen stil tablica baze podataka	& Nuić & 13.11.2020 		\\[3pt] \hline 
			
			0.19 & Dijagrami razreda & Nuić & 13.11.2020 		\\[3pt] \hline 
			
			\textbf{1.0} & Verzija samo s bitnim dijelovima za 1. ciklus & Nuić & 13.11.2020. \\[3pt] \hline 
			
			1.1 & Dijagram stanja & Paradžik & 10.01.2021. \\[3pt] \hline
			
			1.2 & Dijagram aktivnosti & Paradžik & 11.01.2021. \\[3pt] \hline
			
			1.3 & Ispravak dokumentacije iz 1. ciklusa & Nuić & 13.01.2021. \\[3pt] \hline 
			
			1.4 & Osvježavanje dijagrama baze podataka & Nuić & 14.01.2021. \\[3pt] \hline

			1.5 & Zaključak i budući rad & Paradžik & 14.01.2021. \\[3pt] \hline
			
			1.6 & Dodan dijagram komponenti i dijagram razmještaja & Crnogorac & 14.01.2020. \\[3pt] \hline
			
			1.7 & Dodani dijagrami razreda & Nuić & 14.01.2021. \\[3pt] \hline
			
			1.8 & Dodane upute za pokretanje & Nuić & 14.01.2021. \\[3pt] \hline
			
			\textbf{2.0} & Završna datoteka za predaju & Nuić & 14.01.2021. \\[3pt] \hline 
		\end{longtabu}
	
	
%		\textit{Moraju postojati glavne revizije dokumenata 1.0 i 2.0 na kraju prvog i drugog ciklusa. Između tih revizija mogu postojati manje revizije već prema tome kako se dokument bude nadopunjavao. Očekuje se da nakon svake značajnije promjene (dodatka, izmjene, uklanjanja dijelova teksta i popratnih grafičkih sadržaja) dokumenta se to zabilježi kao revizija. Npr., revizije unutar prvog ciklusa će imati oznake 0.1, 0.2, …, 0.9, 0.10, 0.11.. sve do konačne revizije prvog ciklusa 1.0. U drugom ciklusu se nastavlja s revizijama 1.1, 1.2, itd.}