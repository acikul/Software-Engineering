\chapter{Zaključak i budući rad}
		
		Naša grupa imala je zadatak izraditi aplikaciju koja omogućuje jednostavnije povezivanje sportaša, trenera i iznajmljivača sportskih prostora te taj proces pojednostavljuje i ubrzava. Ideja aplikacije je proizašla od članova naše grupe, zbog toga su svi članovi bili dodatno motivirani da uspješno i kvalitetno izradimo aplikaciju te uz nju napišemo kvalitetnu dokumentaciju. 
		
		Za izradu ove aplikacije odlučili smo se koristiti tehnologije Spring Boot za backend, te React za frontend. Najveći izazov je bio naučiti koristiti navedene tehnologije te povezati frontend i backend. Da bismo lakše prevladali ovaj izazov podijelili smo grupu u 2 manja tima, jedan za frontend te jedan za backend, svaki tim je sadržavao po 3 člana naše grupe. Voditelj naše grupe je vodio oba tima, zbog toga su timovi tijekom cijelog projekta bili cijelo vrijeme na istoj razini te ni jedan tim nije zaostajao za drugim. Ovakav raspored te dobra atmosfera i komunikacija među članovima učinila je rad cijele grupe produktivnijim te s time i olakšala prevladavanje svih izazova i prepreka.
		
		Projekt je bio podijeljen u dvije faze kroz 15 tjedana. Prva faza je uključivala okupljanje tima, smišljanja ideje za aplikaciju, podjelu u timove, dodjelu projektnog zadatka, rad na dokumentaciji i implementaciju osnovnih funkcionalnosti aplikacije. Druga faza je uključivala završetak svih funkcionalnosti backenda i frontenda, testiranje aplikacije te dovršetak dokumentacije i cijelog projekta.
		
		Prilikom izrade ovog projekta naučili smo uz dosada navede tehnologije koristiti sustav za upravljanje izvornim kodom Git te alate TEXstudio i Astah koji služe za pisanje dokumentacije i crtanje dijagrama. Osim navedenih tehnologija naučili smo pisati kvalitetnu dokumentaciju te kvalitno oblikovati kod koji je iznimno važan za daljnu nadogradnju projekta, testiranje i tražanje grešaka u kodu. Možda i najvažnija lekcija naučena prilikom izrade ovog projekta je važnost grupnog rada, ispunjavanje svojih obaveza na vrijeme te suradnja i komunikacijama s kolegama. Sve navedene naučene lekcije su od iznimne važnosti za naš daljni razvoj i uspješnost u struci.
		
		
		Za brže i kvalitetnije ostvaranje projekta potrebno je imati već određeno iskustvo u korištenim tehnologijama te iskustvo u sličnim projektima. Osim toga, jako je važno raditi na što manje projekata paralelno, na taj način imali bi više vremena i veći fokus za naš.
		
		Naša aplikacija "FringillaSport" mogla bi se nadograditi na brojne načine. Jedna od ideja za dodatne funkcionalnosti je ocjenjivanje sportaša, trenera te prostora za sport, sportaši bi nakon sudjelovanja na nekom događaju mogli ocijeniti ostale sportaše, trenera te prostor za sport s ciljem kvalitetnije usluge aplikacije i informacijama o kvaliteti sportskih događaja.
		
		
		